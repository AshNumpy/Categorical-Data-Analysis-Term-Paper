\documentclass{article}
\usepackage{graphicx}
\usepackage[T1]{fontenc}
\usepackage{hyperref}
\usepackage{caption}
\usepackage{times}
\usepackage{amsmath}

\title{Kategorik Veri Çözümlemesi}
\author{Ramazan Erduran}
\date{May 2023}

\begin{document}
\begin{titlepage}
    \begin{center}        
        \vspace*{1cm}
        \Huge
        \textbf{Kategorik Veri Çözümlemesi Proje Künyesi}

        \vfill
        
        \Large
        \textit{Ramazan Erduran ~ 21821809 \\
        İlkay Şafak Baytar ~ 21935712 \\
        İhsan Tutak ~ 21822178
        }\\
        
        \vspace{50pt}
        Hacettepe Üniversitesi \\
        İstatistik \\
        
    \end{center}
\end{titlepage}

\newpage
\section{Projenin Adı}
Fen Fakültesi Öğrencilerinin İkili İlişkilerinin İncelenmesi

\section{Proje Açıklaması}
Projenin amacı, Hacettepe Üniversitesi Fen Fakültesi öğrencilerinin ikili ilişkilerdeki davranışını incelemek.

Proje sadece Fen Fakültesi öğrencilerini kapsamaktadır. Projenin Hedefi ise Fen Fakültesi öğrencilerinin ikili ilişkilerdeki davranışlarını inceleyip yorumlamak.

\section{Proje Yürütücüsü}
Proje yürütücüsü Hacettepe Üniversitesi İstatistik Bölümü Kategorik Veri Çözümlemesi Dersi sorumlusu Prof. Dr. Serpil Aktaş Altunay. İletişim adresleri;
\begin{enumerate}
    \item Web Sayfası: \href{https://avesis.hacettepe.edu.tr/spxl}{https://avesis.hacettepe.edu.tr/spxl}
    \item E Posta: \href{mailto:spxl@hacettepe.edu.tr}{spxl@hacettepe.edu.tr}
    \item İş Telefonu: +90 312 297 7930
\end{enumerate}

\section{Proje Ekibi}
Ramazan Erduran, İlkay Şafak Baytar, İhsan Tutak

\section{Proje Başlangıç ve Bitiş Tarihleri}
\begin{itemize}
    \item Proje Başlangıç Tarihi: 15.04.2023
    \item Proje Bitiş Tarihi: 04.06.2023 
\end{itemize}

\section{Proje Bütçesi}
Projeye ayrılan özel bir bütçe bulunmamaktadır.

\section{Kurum}
Projenin Bağlı Olduğu Kurum Hacettepe Üniversitesi'dir.

\section{Proje Süreci ve Metodoloji}
Proje sürecinde veriler tabakalı örnekleme yöntemi ile örnekleme modeli kurularak anket yöntemi ile toplanmıştır.
\begin{itemize}
    \item Anket bağlantısı: \href{https://docs.google.com/forms/d/e/1FAIpQLSf7CBA1tg89lFSdAl9VRSa4vjmv8COCJtRDvQmYa0n0l72JUA/viewform?usp=sf_link}{Anket}
\end{itemize}
Verilerin analizi süreci 5 başlıkta incelenip metodoloji buna göre geliştirilmiştir. Başlıklar:
\begin{enumerate}
    \item Tek değişkenli görselleştirme.
    \item RxC çözümlemesi & uyum analizi
    \item ODDS oranı ve önem kontrolü
    \item OxN ve NxO çözümlemesi
    \item En iyi modelin belirlenmesi
    \item NxNxO, NxOxO ya da OxOxO modellerinden birinin çözümlemesi ve serbestlik derecesinin çıkarsamasının gösterilmesi
\end{enumerate}

\section{Proje Kaynakları}
Proje süresince Python ve R programlama dilleri kullanılarak analizler gerçekleştirilmiştir. Raporlama ise LaTex belge düzenleme biçimiyle hazırlanmıştır. 
Analiz süresince yararlanılan kaynaklar raporlamanın "Kaynakça" bölümünde belirtilmiştir.

\section{İletişim Bilgileri}
\begin{itemize}
    \item Ramazan Erduran: \href{mailto:ramazan.erduran@outlook.com.tr}{ramazan.erduran@outlook.com.tr}
    \item İlkay Baytar: \href{mailto:ilkaybaytar@gmail.com}{ilkaybaytar@gmail.com}
    \item İhsan Tutak: \href{mailto:ihsan.tutak@gmail.com}{ihsan.tutak@gmail.com}
\end{itemize}

\end{document}