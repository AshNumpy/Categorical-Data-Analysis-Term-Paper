\documentclass{article}
\usepackage{graphicx}
\usepackage[T1]{fontenc}
\usepackage{hyperref}
\usepackage{caption}
\usepackage{times}
\usepackage{amsmath}

\title{Kategorik Veri Çözümlemesi}
\author{Ramazan Erduran}
\date{May 2023}

\begin{document}
\begin{titlepage}
    \begin{center}
        \includegraphics[width=0.5\textwidth]{hacettepe_logo.png}
        
        \vspace*{1cm}
        \Huge
        \textbf{Kategorik Veri Çözümlemesi Dönem Ödevi}

        \vfill
        
        \Large
        İstatistik \\
        Hacettepe Üniversitesi \\
        Ramazan Erduran \\
        05.05.2023
        
    \end{center}
\end{titlepage}

\newpage

\section{Araştırmanın Hakkında}
\subsection{Araştırmanın Başlığı}
Fen Fakültesi Öğrencilerinin İkili İlişkilerinin İncelenmesi.

\subsection{Araştırmanın Amacı}

\subsection{Örnekleme Planlaması}
Araştırmalar sonucunda Hacettepe Üniversitesi'nin yayınladığı bilgilere dayanarak öğrenci, öğretim üyesi oranı, Tablo. \ref{tab:Öğrenci:Öğretim Üyesi Oranları} 'deki veriler bulunmuştur.

\begin{table}[h]

    \centering
    \caption{Öğrenci:Öğretim Üyesi Oranları}
    \label{tab:Öğrenci:Öğretim Üyesi Oranları}

    \begin{tabular}{|p{2.1cm}|p{2.1cm}|p{2.1cm}|p{2.1cm}|p{2.1cm}|}
        \hline
         & Tıp ve Sağlık & Fen ve Mühendislik & Sosyal ve Beşeri Bilimler & Toplam \\
        \hline
        Öğretim Üyesi'ne Oranı & 16:1 & 35:1 & 46:1 & 30:1 \\
        \hline
        Öğretim Elemanı'na Oranı & 6:1 & 22:1 & 23:1 & 14:1 \\
        \hline
    \end{tabular}
    \caption*{\footnotesize Kaynak:\url{https://www.hacettepe.edu.tr/ogretim/sayilarla_ogretim}}
\end{table}

Ayrıca her bölümün kendi sayfasında öğretim üyelerinin kesin sayıları olduğundan toplam öğrenci sayısının hesaplanmasında belirtilen varsayımlar kullanılmıştır.

\begin{table}[h]
    \centering
    \caption{Bölüm Bazlı Akademik Personel Sayısı}
    \label{tab:Bölüm Bazlı Veriler}
    
    \begin{tabular}{|c|c|c|}
         \hline
         Bölüm & Öğretim Elemanı Sayısı & Tabaka Ağırlığı\\
         \hline
         İstatistik & 26 & 0.13 \\
         Aktüerya & 6 & 0.03 \\
         Matematik & 41 & 0.20 \\
         Kimya & 51 & 0.25 \\
         Biyoloji & 81 & 0.39 \\
         \hline
         Toplam & 205 & 1.00 \\
         \hline
    \end{tabular}
\end{table}

\clearpage

\begin{table}[h]
    \centering
    \caption{Bölümlere Göre Öğrenci Sayıları}
    \label{tab:Bölüm bazlı öğrenci sayıları}
    \begin{tabular}{|c|c|}
        \hline
        Bölüm & Öğrenci Sayısı\\
        \hline
        İsatistik & 572 \\
        Aktüerya & 132 \\
        Matematik & 902 \\
        Kimya & 1122 \\
        Biyoloji & 1782 \\
        \hline
        Toplam & 4510 \\
        \hline
    \end{tabular}
\end{table}

Tablo \ref{tab:Bölüm bazlı öğrenci sayıları}'deki öğrenci sayıları, Tablo \ref{tab:Öğrenci:Öğretim Üyesi Oranları} ve Tablo \ref{tab:Bölüm Bazlı Veriler}'den hareketle hesaplanmıştır. 


\section{Örnekleme Planlaması}
Tablo \ref{tab:Bölüm bazlı öğrenci sayıları}'den hareketle aşağıdaki formül uygulanarak örneklem sayısı elde edilmiştir:
$$
n = N \times \frac{Z^2 \times p \times (1-p)}{d^2 \times (N-1) \times Z^2 \times p \times (1-p)}
$$

Formülde kullanılan parametreler;
\begin{itemize}
    \item Z = 1.96 (\%95 Güven Düzeyinde)
    \item N = 4510
    \item d = 10
    \item p = 0.5
\end{itemize} şeklindedir. Formül uygulandıktan sonra $n=68.34 \approx 68$ olarak bulunmuştur. Bulunan örneklem sayısı bölümlerin tabakalı ağırlıklarına göre dağıtıldığında toplanması gerekilen örneklem sayısı Tablo \ref{tab:ağırlıklandırılmış örneklem}'deki gibi elde edilmiştir.

\begin{table}[h]
    \centering
    \caption{Bölümlere Göre Örneklem Sayısı}
    \label{tab:ağırlıklandırılmış örneklem}
    \begin{tabular}{|c|c|}
         \hline
         Bölüm & Örneklem Sayısı \\
         \hline
         İstatistik & 9 \\
         Aktüerya & 2 \\
         Matematik & 14 \\
         Kimya & 17 \\
         Biyoloji & 25 \\
         \hline
         Toplam & 68 \\
         \hline
    \end{tabular}
\end{table}

\section{Kaynakça}
\begin{enumerate}
    \item \href{https://www.hacettepe.edu.tr/ogretim/sayilarla_ogretim}{Hacettepe Üniversitesi fakülte bazlı numerik genel bilgiler}
    \item \href{https://stat.hacettepe.edu.tr/tr/menu/ogretim_uyeleri-118}{Hacettepe Üniversitesi İstatistik Bölümü akademik personel verileri}
    \item \href{https://aktuerya.hacettepe.edu.tr/pers_ou.php}{Hacettepe Üniversitesi Aktüerya Bölümü akademik personel verileri}
    \item \href{https://mat.hacettepe.edu.tr/akademik_personel.html}{Hacettepe Üniversitesi Matematik Bölümü akademik personel verileri}
    \item \href{https://chem.hacettepe.edu.tr/tr/menu/ogretim_uyeleri-10}{Hacettepe Üniversitesi Kimya Bölümü akademik personel verileri}
    \item \href{https://biology.hacettepe.edu.tr/tr/akademik_personel-226}{Hacettepe Üniversitesi Biyoloji Bölümü akademik personel verileri}
\end{enumerate}

\end{document}

